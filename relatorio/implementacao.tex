\section{Implementa��o}

O trabalho consiste em identificar dinamicamente os par�metros do sistema para
plantas de primeira, segunda e terceira ordem. Nesta se��o, iremos descrever a
implementa��o dos algoritmos do gradiente normalizado e \textit{least-square},
em MATLAB.

A primeira etapa da implementa��o � definir as condi��es iniciais, par�metros do
sistema a serem identificados, fun��o de transfer�ncia da planta, filtro e sinal
de entrada.

As condi��es iniciais do sistema s�o os valores para $y(0)$ e $\theta_0$.
Consideremos uma planta est�vel de terceira ordem de forma a exemplificar com o
sistema mais geral deste trabalho:

\begin{gather}
P(s)[y](t) = Z(s)[u](t),\\
\nonumber P(s) = (s+0.5)^3 = s^3 + 1.5 s^2 + 0.75s + 0.125,\\
\nonumber Z(s) = (s+0.3)^3 = s^2 + 0.6 s + 0.09,\\
\nonumber \vec{pz} = \left[z_2, z_1, z_0, p_2, p_1,p_0\right]^\intercal=
\left[1 \quad 0.6 \quad 0.09 \quad 1.5 \quad 0.75 \quad 0.125\right]^\intercal,\\
\nonumber u(t) = r(t) = \mathrm{dc} + \mathrm{sin}(w_1t) + \mathrm{sin}(w_2t) +
\mathrm{sin}(w_3t).
\end{gather}

O sinal de refer�ncia deve possuir pelo menos seis frequ�ncias, pois h� seis
par�metros desconhecidos (Se��o 3.5 - Tao). Al�m disso, o sistema �
dito com excita��o persistente se $u(t)$ n�o possui frequ�ncias do regressor
$\phi(t)$. 

Considere um filtro de terceira ordem:

\begin{gather}
\Lambda(s) = (s+1)^3 = s^3+3s^2+3s+1,\\
\nonumber \vec{\lambda} = \left[3 \quad  3 \quad  1\right]^\intercal.
\end{gather}

O vetor de par�metros � definido como:

\begin{gather}
\theta^* =
\left[z_0 \quad z_1 \quad z_2 \quad \lambda_0-p_0 \quad \lambda_1-p_1 \quad \lambda_{2}-p_{2}\right]^\intercal = \\
\nonumber \theta^* = \left[\textrm{flip}(\vec{pz}(1:3))  \quad 
\textrm{flip}(\vec{\lambda}-\vec{pz}(4:6))\right]^\intercal = \\
\nonumber \theta^* = \left[0.09 \quad 0.6 \quad 1 \quad 0.875 \quad 2.25 \quad 1.5\right]^\intercal.\\
\end{gather}

As equa��es do filtro podem ser descritas como um sistema no espa�o de estados:

\[
\begin{cases}
\dot{\omega}_{11} = \omega_{12}, \\
\dot{\omega}_{12} = \omega_{13}, \\
\dot{\omega}_{13} = u - \textrm{flip}(\vec{\lambda})^\intercal\omega
\end{cases}
\]

\[
\begin{cases}
\dot{\omega}_{21} = \omega_{22}, \\
\dot{\omega}_{22} = \omega_{23}, \\
\dot{\omega}_{23} = y - \textrm{flip}(\vec{\lambda})^\intercal\omega
\end{cases}
\]

Resumidamente:

\begin{gather}
d\omega_1 = \left[\omega_1(2:3) \quad y -
\textrm{flip}(\vec{\lambda})^\intercal \omega_1 \right]^\intercal,\\
\nonumber d\omega_2 = \left[\omega_2(2:3) \quad y -
\textrm{flip}(\vec{\lambda})^\intercal \omega_2 \right]^\intercal,\\
\phi = \left[\omega_1 \quad \omega_2 \right].
\end{gather}

Podemos descrever a sa�da estimada e real do sistema pelos filtros:

\begin{gather}
y = \theta^{*\intercal}\phi,\\
\hat{y} = \theta^\intercal \phi, \\
\epsilon = \hat{y}-y = \tilde{\theta}^\intercal \phi.
\end{gather}

Finalmente, descrevemos a atualiza��o de $\theta$, os par�metros estimados,
conforme o gradiente normalizado ou pelo m�todo \textit{least-square}:

\[
\textrm{Gradiente} = 
\begin{cases}
m^2 \leftarrow 1 + \phi^\intercal\phi\\
\dot{\theta} \leftarrow \frac{-\Gamma\phi\epsilon}{m^2}
\end{cases}
\]

\[
\textrm{Least-square} = 
\begin{cases}
m^2 \leftarrow 1 + \phi^\intercal P\phi\\
\dot{\theta} \leftarrow \frac{-P\phi\epsilon}{m^2}\\
\dot{P} \leftarrow \frac{-P\phi\phi^\intercal P}{m^2}
\end{cases}
\]
